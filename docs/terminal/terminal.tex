Linux es un sistema operativo open-source. Este posee una herramienta que se le conoce como la terminal. La terminal nos permitira mover y hacer cambios dentro del sistema operativo.

La terminal usa el lenguaje de programación bash para comunicarse con la terminal. Ademas de poder usar este lenguaje, tambien vienen comandos programados previamente para poder usar. En esta sección aprenderemos a usar estos comandos de la terminal, más no como programar en bash (sin contar alguna semantica basica).

En general, estos comandos tienen un monton de opciones ademas de su función especifica. Podemos variar el comportamiento utilizando lo que se les conoce como 'flag', las cuales permiten modificar el comportamiento que queremos que tenga el comando, parecido como a modificar las opciones de un programa. Hay un monton de flags para varios comandos, generalmente denotados de la forma \texttt{--<flag>} (aunque puede variar), y no da el espacio para explicarlos todos con profunidad. Por eso, si se necesita saber que hace un comando o todas las opciones ('flags') que este tienen, se puede usar el comando manual (man).

\begin{lstlisting}[language=bash, style=terminal, caption={man: (man)ual para comandos en Linux}]
user@hostname:~\$ man <command>
\end{lstlisting}

Por ultimo, se puede ver la sección de lstlisting para ver todos los comandos en caso que necesite revisar alguno, o guias las cuales dejaremos mencionadas en su sección respectiva.

\section{Movimiento en terminal}
Antes de empezar, debemos reconocer la estructura de los archivos en Linux. Linux estructura sus archivos mediante FHS, que signifca 'filesystem Hierarchy System'. 

Eso significa que los archivos estan de una forma jerarquia. Todo archivo es hijo de un padre (Es subcarpeta de otra). La unica carpeta que no tiene padre es la carpeta principal que se le conoce como 'root' (raiz). Desde la raiz se encuentran varios archivos que permiten el funcionamiento del sistema operativo. Aqui se ejemplifica la estructura de manera visual

\begin{lstlisting}[language=bash]
	 /
	 |-- bin/
	 |-- boot/
	 |-- dev/
	 |-- etc/
	 |-- home/
	 |-- lib/
	 |-- mnt/
	 |-- opt/
	 |-- tmp/
	 |-- usr/
	 ..  
\end{lstlisting}

Cuando nosotros abrimos la terminal, esta abrira dentro de una ubicación preterminada. Para saber en que archivo estamos de la terminal, se puede usar el comando pwd, que indica (p)rint (w)orking (d)irectory (Imprimir directorio/carpeta actual).

\begin{lstlisting}[language=bash, style=terminal, caption={pwd: Ver ubicación actual}]
user@hostname:~\$ pwd
\end{lstlisting}

\begin{ejer}
	 Al abrir la terminal de linux, obtenga el directorio en donde empieza al iniciar la terminal. 
\end{ejer}

Tras haber hecho el ejercicio, se dara cuenta que la ubicación es de la forma 'home/<username>'. Cada vez que abrimos la temrinal se abre en la carpeta home asociada al usuario ingresado.

Si ademas queremos saber que carpetas se encuentran dentro de la que nos encontramos, se puede usar el comando \texttt{ls}, que muestra los archivos dentro de nuestra carpeta.

\begin{lstlisting}[language=bash, style=terminal, caption={ls: Lista de archivos en el directorio actual}]
user@hostname:~\$ ls
\end{lstlisting}

Notemos que si usamos este comando no mostrara aquellas carpetas ocultas por el sistema. Las carpetas ocultas empiezan con el caracter '.', por lo que la carpeta '.fold' no se mostraria usando el comando ls. Para hacer que la muestre, debemos usar la flag \texttt{-a}, que indica que mostremos todo lo que contiene la carpeta, includio archivos ocultos.

\begin{lstlisting}[language=bash, style=terminal, caption={ls: Lista de archivos con archivos ocultos}]
user@hostname:~\$ ls -a
\end{lstlisting}

Por ultimo si queremos ver que contiene una carpeta que no es la carpeta actual, se puede indicar el archivo que queremos ver indicando su ubicación dentro del sistema.

\begin{lstlisting}[language=bash, style=terminal, caption={ls: Lista de archivos de un directorio especifico}]
user@hostname:~\$ ls <filepath> 
\end{lstlisting}

\subsubsection{Ubicación de archivo (filepath)}

El \textbf{filepath} es la ubicación de un archivo. Esta puede ser indicada de dos maneras distintas. Considere la siguiente forma de los archivos.

\begin{lstlisting}[language=bash]
	/
	|-- bin/
	|-- home/
	|	| -- user2
	|	|	| -- Folder3
	|	| -- user1
	|	|	| -- Workspace
	|	|	| -- Downloads
	|	|	| -- Documents (*)
	|	|	|	| -- Folder1
	|	|	|	| -- Folder2
	..
\end{lstlisting}

Digamos que la terminal se encuentra en la carpeta Documents, que tiene las siguientes caracteristicas.

\begin{itemize}
	\item Folder1 y Folder2 son hijas de Documents
	\item Downloads y Workspace son hermanas de Documents
	\item user1 es padre de Documents
\end{itemize}

Si queremos indicar al comando \texttt{ls} (o cualquier otro comando) la ubicación del archivo, debemos indicarle la ubicación. Hay dos maneras de indicar esto. Consideremos que queremos ir hacia la carpeta 'Folder1'.

\begin{itemize}
	\item \textbf{Relative Path (Ruta relativa)}: Empieza desde la carpeta actual. Si quisieramos movernos hacia Folder1, entonces debemos indicar al comando ls de la forma \texttt{Folder1}, ya que se encuentra dentro de la carpeta actual.
	\item \textbf{Absolute Path (Ruta absoluta)}: Esta empieza desde la carpeta root (que es indicada mediante un /). En este caso, se indicaria de la forma \texttt{/home/user1/Documents/Folder1}
\end{itemize}

Ambas maneras indican el mismo camino hacia el archivo, solo que en distinto lenguaje. Ahora, en vez de querer ver los contenidos de Folder1, queremos ver los contenidos de la carpeta 'Folder3'.

\begin{itemize}
	\item \textbf{Relative Path}: Deberemos indicarle que queremos ver una carpeta superior a la que nosotros. Podemos indicar con \texttt{..} que queremos subir una escala en la jerarquia, por lo que la ruta se veria de la forma \texttt{../../user2/Folder3}, en donde subimos jerarquia (de Documents a user1), subimos nuevamente (de user1 a home) y accedemos a las carpetas dentro de este (user2 y luego Folder3)
	\item \textbf{Absolute Path}: Esta empieza de la carpeta root, por lo que unicamente indicamos la ruta. Queda de la forma \texttt{/home/user2/Folder3}
\end{itemize}

Dependiendo del contexto, una sera mas facil que escribir de la otra, por lo que es util saber que existen ambas formas de escritura. Ademas, el filepath puede describir la ubicación tanto de archivos como carpeta, asi que es usado por otros comandos para modificar archivos.

Por ultimo, si queremos movernos a otra carpeta en la terminal, podemos usar el comando \texttt{cd}. Recibe tambine coomo input el filepath, y movera la terminal a la ubicación indicada.

\begin{lstlisting}[language=bash, style=terminal, caption={cd: Cambiar de directorio}]
user@hostname:~\$ cd <filepath>
\end{lstlisting}

Si queremos movernos a una carpeta, indicamos su nombre. Si queremos subir al directorio padre, tambien se puede indicar mediante \texttt{..}. Es cosa de entender el filepath.

Por ultimo, si se quiere volver a la ubicación preterminada de inicio, utilice \texttt{cd} sin ningun filepath.


\section{Manejo de archivo}
\subsection{Manejo de archivos}

Para la creación de archivos podemos utilizar varias maneras. Uno de los comandos mas usados es \texttt{touch}. Este comando cuando se le indica un archivo ya creado cambiara su ultima fecha de acceso (timestamp), pero si el archivo no existe lo creara.

\begin{lstlisting}[language=bash, style=terminal, caption={touch: Creación de un archivo o cambiar timestamp}]
user@hostname:~\$ touch <filename>
\end{lstlisting}

Luego, para remover archivos se utiliza el comando \texttt{rm}, el cual significa (r)e(m)ove. Se indica nombre del archivo a eliminar.

\begin{lstlisting}[language=bash, style=terminal, caption={rm: (R)e(m)over un archivo}]
user@hostname:~\$ rm <filename>
\end{lstlisting}

Si queremos mover uno (o varios) archivos a otra localización, se utiliza el comando mv, que indica (m)o(v)e.

\begin{lstlisting}[language=bash, style=terminal, caption={mv: (m)o(v)er uno (o varios) archivos a una ubicación}]
user@hostname:~\$ mv <filename>... <filepath>
\end{lstlisting}

Un truco util de \texttt{mv} es que tambien nos puede servir para cambiar el nombre del archivo. Dado que el filepath puede indicar tanto carpetas como archivos especificos, podemos indicarle un nombre de archivo nuevo, haciendo entonces que cambie de nombre.

\begin{lstlisting}[language=bash, style=terminal, caption={mv: Cambiar nombre de archivo}]
user@hostname:~\$ mv <filename> <new-filename>
\end{lstlisting}

Otra manera tambien de mover archivos es usar \texttt{cp}. \texttt{cp} crea una (c)o(p)ia del archivo y la deja en la ubicación indicada.

\begin{lstlisting}[language=bash, style=terminal, caption={cp: (c)o(p)iar un archivo}]
user@hostname:~\$ cp <filename> <filepath>
\end{lstlisting}

Por ultimo, si queremos ver los contenidos de un archivo en la terminal, se puede utilizar el comando \texttt{cat}.

\begin{lstlisting}[language=bash, style=terminal, caption={cat: Ver contenidos del archivo}]
user@hostname:~\$ cat <filepath>
\end{lstlisting}
Para editar el archivo, se puede utilizar la propia terminal, o puede usarse el comando \texttt{code} para abrir \texttt{Visual Studio Code} en la carpeta o archivo indicado mediante el filepath. Tambien puede usarse otro editor conocido como \texttt{Vim} o \texttt{Neovim}, que funciona directamente en la terminal y permite movertte a traves de ella sin necesidad de mouse. Se va fuera de los contenidos de esta sección, pero dejamos mencionados para que trabajen su curiosidad.

\subsection{Manejo de carpetas}

Para la creación de carpetas, se utiliza el comando \texttt{mkdir} ((m)a(k)e (dir)ectory). Se le puede indicar un filepath, pero debe existir para que funcione correctramente.

\begin{lstlisting}[language=bash, style=terminal, caption={mkdir: Crear directorio nuevo}]
user@hostname:~\$ mkdir <filepath>
\end{lstlisting}

El comando \texttt{mv} tambien permite mover carpetas. Su uso es similar al indicado en archivos, solo que en vez de indicarle un archivo le indicamos una carpeta. Los contenidos de la carpeta seran trasladados tambien. Por otro lado, \texttt{cp} tambien funcionara con carpetas. Al final, ambos comandos sirven para mover el archivo indicado por el filepath, por lo que puede ser cualquiera de los dos.

Una diferencia importante si con el manejo de archivos es la eliminación de carpetas. Existe el comando \texttt{rmdir} que permite (r)e(m)over un (dir)ectorio/carpeta, pero este unicamente funciona si los elementos de la carpeta se encuentran vacios.

\begin{lstlisting}[language=bash, style=terminal, caption={rmdir: (R)e(m)over un (dir)ectorio}]
user@hostname:~\$ rmdir <filepath>
\end{lstlisting}

Si queremos eliminar una carpeta con los contenidos dentro de esta, podemos volver al comando \texttt{rm}, con la flag de \texttt{-r}. Con esta flag indicamos que queremos eliminar la carpeta y sus contenidos recursivamente. 

\begin{lstlisting}[language=bash, style=terminal, caption={rm: (R)e(m)over un (dir)ectorio con contenidos}]
user@hostname:~\$ rm -r <filepath>
\end{lstlisting}

Este comando de por si es muy peligroso. Dada la estructura de archivos de linux vista anteriormente, como todo nace de la carpeta 'root', entonces es posible que al usar este comando borremos todo el contenido de nuestro sistema operativo (Haciendolo inutilizable), o borrar mas carpetas de las que deseabamos. Ademas, el comando no tiene confirmación, pero puede agregarse la flag \texttt{-i} para que la tenga, haciendolo un poco mas seguro.




\section{Busqueda de archivos}
\input{docs/terminal/busqueda)}

\section{Ejercicios}
\section{Ejercicios Adicionales}

(Nota, incluyen algunos que utilizan el comando sed para cambiar las palabras. Ver despues el agregar este comando a la lista de comandos necesarios para aprender linux)

\begin{ejer}
    Un amigo suyo le entrego una carpeta con canciones de su banda favorita, Hesse Kassel, tanto en formato mp4 como mp3. Usted desea unicamente los archivos mp3. Haga un comando que remueva todos los archivos .mp4 de la carpeta actual.

    Pista: Puede hacer algo parecido a \texttt{ls <patron>} para decirle a \texttt{rm} que remueve aquellos que coincidan con un patron.
\end{ejer}

\begin{ejer}
    El comando ps aux retorna una lista con todos los procesos que se ejecutan en el momento. Si se sabe que todo proceso que usa la cpu contiene 'cpu' en el nombre, utilize el comando ps para mostrar en la terminal todos los procesos ejecutandose en la cpu.

    Pista: Utilice la \texttt{pipeline} para este proposito.
\end{ejer}

\begin{ejer}
    En la carpeta downloads se instalo un ejecutable de un compilador llamado typst. Usted quiere que este ejecutable pueda ejecutarse desde cualquier ubicación usando la terminal. Escriba un comando que realice esto, asumiendo que se encuentra en la misma ubicación que el ejecutable.

    Pista: Analice los resultados del ejercicio 7 para poder resolver este ejercicio.
\end{ejer}

\begin{ejer}
    Usted investigando aprendio un comando muy util llamado \texttt{xargs}. Este permite transformar cualquier linea de texto y entregar por input como separado cualquier elemento que este con un espacio entre medio. Por ejemplo, \texttt{xargs "x1 x2"} retornaria \texttt{x1} y \texttt{x2} por separado.

    Utilice este comando para mover de una carpeta todos los archivos que tengan el formato \texttt{.c}.
\end{ejer}

\begin{ejer}
    Benjito es un compañero suyo que le gusta mucho hacer documentos en latex, pero siempre está revisando documentos antiguos para recordar la configuración que le gusta para cada tipo de documento. Benjito se aburrió de copiar manualmente y se le ocurrió hacer una template que la guardará en ~/templates.
    
    Escriba un comando que le permita a benjito poder copiar el contenido de template en una carpeta con nombre personalizado
\end{ejer}

\begin{ejer}
    Usted esta realizando un codigo en C++ que utiliza archivos header (.h). Leyendo aprendio que habia una mejor forma para indicar los headers de C++, la cual es .hpp, ya que .h es para indicar headers de C. Realice un comando en la terminal que le permita solucionar su error, asumiendo que todos los .h se encuentran en la carpeta actual.

    Pista: Investigue el comando \texttt{rename} para cumplir este proposito.
\end{ejer}

\begin{ejer}
    Un compañero suyo le paso casos de prueba para un codigo que estaban haciendo. En este, tiene que verificar las rimas de los versos de un .txt llamado estrofa.txt. Sin embargo, se dio cuenta que su compañero escribio todas las ´a´ como ´4´, haciendo que palabras como armadura terminen siendo 4rm4dur4. Realice desde la terminal un comando que arregle este error de tipeo.
\end{ejer}

