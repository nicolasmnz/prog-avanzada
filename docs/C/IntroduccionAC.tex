\section{Tipado Estatico}

Hemos visto como la funcion de `main()` dentro del codigo tiene un posee un `int` asociado a el. Esto es porque estamos declarando que valor va a retornar la función.

Previamente, en Python, las funciones no debian de indicar que retorna, lo que se conoce como Tipado Dinamico. A diferencia de Python, en C debemos declarar que tipo de datos va a retornar la función. Esto se conoce como Tipado Estatico. C es un lenguaje con Tipado Estatico.

Esto no solo se cumple para las funciones, si no que tambien debemos de declarar que tipo de valor guardan las variables. Los tipos basicos de C son:

\begin{itemize}
    \item \textbf{int: } Valor numerico que guarda numeros enteros.
    \item \textbf{char: } Contiene unicamente un caracter
    \item \textbf{float: } Valor numerico que guarda numeros reales.
\end{itemize}

C realiza Tipado Estatico para poder saber la cantidad de memoria que sera utilizado por una variable o retorno de esta. Ademas, operadores entre distintos tipos de valores no son permitidos. 

Para ver la diferencia de memoria entre los tipos de datos, realizaremos un ejercicio que necesita saber de los siguiente operadores.

\begin{lstlisting}[language=c,style=customc,caption={Operador sizeof}]

#include <studio.h>
int main(){
    int x; // Variable con tipo de dato int
    printf("Cantidad de memoria utilizada: %d", sizeof(x));
}

\end{lstlisting}

En este ejemplo, el operador sizeof entrega cuanta memoria utiliza la variable x.

\begin{ejer}
    Cree un programa en C llamado `SizeofTypes` que imprimia por pantalla los datos de tipo `int`. `float`, y `double`. Comparelos.
\end{ejer}

Para imprimir numeros por pantalla, use el comando `printf("Cantidad de memoria utilizada: \%d", sizeof(x));`, en donde x corresponde a las variables con cada uno de los tipos. No se preocupe del funcionamiento de esta funcion, se explicara en la siguiente sección.

\section{Input y Output}

En la sección anterior vimos un nuevo uso de la función `printf()`, que permitia mostrar por pantalla el valor retornado por la operación `sizeof`. Ahora explicaremos en mas detalle el input y output.

En C, para realizar input se utiliza el comando `scanf`.

\begin{lstlisting}[language=c,style=customc,caption={Función scanf (Input en C)}]

#include <studio.h>
int main(){
    int x; // Variable con tipo de dato int
    scanf("%d", &x);
}
\end{lstlisting}

La función scanf toma un input desde la terminal y se lo pasa a una variable.
\begin{itemize}
    \item El primer parametro de la función, \" \%d \" , se le conoce como un especificador de formato. Estos especificadores indican el tipo de dato que se va a recibir.
    \item El segundo parametro corresponde a la variable en donde guardaremos el tipo de dato. \& indica su dirección en memoria, que se explicara en mas detalle mas adelante.
\end{itemize}

\begin{ejer}
    Cree un programa en C llamado `SumaNumeros` que reciba por pantalla dos numeros,  y que se muestre por pantalla el resultado de sumarlos. Considere que siempre se recibira un numero entero.
\end{ejer}

Los especificadores de formato permiten identificar el tipo de dato para mostrarlo correctamente en las funciones de `scanf` y `printf`. Los especificadores de formato son los siguientes:

\begin{lstlisting}[language=c,style=customc,caption={Especificadores de formato para scanf}]

#include <studio.h>
int main(){

    // ... Se declaran las variables previamente
    
    scanf("%d", &a); // Input del tipo digito
    scanf("%c", &b); // Input del tipo caracter
    scanf("%s", &c); // Input del tipo string
    scanf("%p", &d); // Input del tipo puntero
    scanf("%l", &e); // Input del tipo unsigned long 
}
\end{lstlisting}

Estos especificadores de formato tambien pueden usarse para mostrar por pantalla usando `printf`.

\begin{ejer}
    Cree un programa en C llamado `esPar`, el cual recibe un numero por pantalla, el cual muestre un caracter por pantalla, `T` en caso de ser par, y `F` en caso de no serlo. (Pista: Utilice el operador \%, aprendido de python, para este ejercicio).
\end{ejer}
