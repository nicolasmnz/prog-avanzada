Las variables en C se identifican mediante su nombre y tipo. Previamente en Python, podiamos ingresar cualquier tipo de dato a una variable, segun el contexto que necesitmos. En C esto no es posible, se le necesita indicar especificamente que tipo de dato va a guardar la variable. Esto se le conoce como \textbf{Tipado Estatico}

En el siguiente codigo se ve como declarar variables

\begin{lstlisting}[language=c,style=customc,caption={Sentencias de Declaración de Variables}]
int main(){
	int x; // Variable con nombre x que guarda datos del tipo entero
	float y; // Variable con nombre y que guarda datos del tipo flotante (decimales)
	char c; // Variable con nombre c que guarda datos del tipo caracter
}
\end{lstlisting}

Hay muchos tipos de datos en C. Uno se podria preguntar el porque debemos de declarar los tipos de datos previo al uso de una variable. Resumido, C asigna una cantidad de memoria especifica a cada variable para asi poder ser eficiente. Las variables de tipo \texttt{int} pueden guardar mas datos que una variable del tipo \texttt{char}. La diferencia viene en la cantidad de bytes.
Un byte contiene 8 bits, y cada bit indica el numero $0$ o el numero $1$. Mientras mas bits se tenga, mas valores se puede representar. Los valores vienen de la forma $2^{\text{bits}}$, con 8 bits pueden representarse $2^8 = 128$ valores. Con 4 bytes, tendriamos 32 bits, y por tanto $2^32 = 4294967296$.

Podemos entonces notar que cada tipo nos permitira identificar el rango de valores que puede contener la función. Aqui se tiene una tabla de los tipos de datos mas comunes

\begin{table}
	\centering
	\caption{Tipos de Datos en C}
	\begin{tabular}{|l|c|}
		\hline
		\texttt{char} & Para datos caracteres. Utiliza un byte. \\
		\texttt{int} & Para datos de tipo numerico entero. Utiliza 4 bytes. \\
		\texttt{float} & Para datos de tipo flotante (decimales). Utiliza 4 bytes \\
		\hline
	\end{tabular}
\end{table}

Para los tipos de datos numericos \texttt{int}, se puede expandir su memoria utilizando el termino \texttt{long}. Un dato que utilize este tipo representa numeros enteros igualmente, pero con mas bytes para mayor precision. Lo mismo podemos hacer para valores del tipo flotante, utilizando el tipo \texttt{double}.

Dependiendo del sistema, sera una cantidad o otra. Podemos saber cuanta cantidad de bytes es utilizada por una variable utilizando el operador \texttt{sizeof}.

Por ejemplo, este programa mostrara por pantalla la cantidad de bytes de los tipos de datos declarados previamente.

\begin{lstlisting}[language=c,style=customc,caption={Mostrar tamaño mediante sizeof}]
int main(){
	int x; // Variable con nombre x que guarda datos del tipo entero
	float y; // Variable con nombre y que guarda datos del tipo flotante (decimales)
	char c; // Variable con nombre c que guarda datos del tipo caracter

	printf("Tamaño variable int: %d", &x);
	printf("Tamaño variable float: %d", &y);
	printf("Tamaño variable char: %d, &c");
}
\end{lstlisting}

Utilize el comando de printf para ver cuanta memoria es utilizada por los tipos \texttt{long} y \texttt{double}. (Para propositos de este ejercicio, copie el printf del codigo anterior, la explicación del funcionamiento del scanf se vera mas adelante).

