\section{Condicionales}

En este laboratorio se veran formas de control dentro del lenguaje de C.

\textbf{Condición} corresponde a una expresion la cual sera evaluada. Esta expresion puede estar compuesta de distintas maneras. En C, el valor $0$ indica \texttt{false}, y cualquier valor distinto de este como \texttt{true}.

La condición puede estar compuesta de \textbf{expresiones relacionales}, que comparten similitud con las vistas en python. Estas corresponden a las siguientes.

\begin{table}[H]
    \centering
    \begin{tabular}{|l|p{5.5cm}|p{4.5cm}|}
        \hline
        \textbf{Expresion Relacional} & \textbf{¿Que hace?} & \textbf{Ejemplo}\\
        \hline
        \texttt{<} & Indica menor que. Puede indicar mayor que si se usa su variación \texttt{>} & \texttt{x < y} devolvera 1 si $x$ es menor que $y$.\\
        \hline
        \texttt{<=}& Indica menor igual que. Puede indicar mayor igual que si usa su variación \texttt{>=}& \texttt{x <= y} devolvera 1 si $x$ es menor o igual que $y$.\\
        \hline
        \texttt{==} & Indica si ambos elementos son iguales. &\texttt{8 == 5+3} retornara 1, mientras que \texttt{1 == '1'} retornara 0. \\
        \hline
        \hline
    \end{tabular}
    \caption{Listado de Expresiones relacionales}
\end{table}

Por otro lado, podemos juntar expresiones relaciones para hacer \textbf{expresiones logicas}. En python, se utilizaban las palabras claves \texttt{and} o \texttt{or}, pero en C no se denotan de esta manera.

\begin{table}[H]
    \centering
    \begin{tabular}{|l|p{5.5cm}|p{4.5cm}|}
        \hline
        \textbf{Expresion Logica} & \textbf{¿Que hace?} & \textbf{Ejemplo}\\
        \hline
        \texttt{\&\&} & Sirve para indicar un \texttt{and} & \texttt{x>1 \&\& x<5} indica que $x$ sea mayor que uno y menor que 5.\\
        \hline
        \texttt{||} & Sirve para indicar un \texttt{or} & \texttt{x > 0 || y > 0} indica que $x$ sea mayor que 0 o $y$ sea mayor que 0.\\
        \hline
        \texttt{!} & Sirve para indicar un \texttt{negado} & \texttt{!(x)} indica el negado del valor de x. Si es true, lo convierte en false. \\
        \hline
    \end{tabular}
    \caption{Listado de Expresiones Logicas}
\end{table}

\subsection{If}

En C, el condicional \texttt{if} permite hacer que ande un bloque de codigo dependiendo de una condición. Continua con la siguiente syntax.

\begin{lstlisting}[language=c,style=customc,caption={Bloque If}]
if(condicion){
    // Codigo
}
\end{lstlisting}

\begin{ejer}
    Haga un codigo en C llamado \texttt{esPar}, que reciba un numero por terminal y escriba un SI en caso que sea verdad. En otro caso, que no haga nada.
\end{ejer}

Podemos hacer este bloque condicional mas poderoso utilizando una expresion \texttt{else}, que permite agregar un bloque de codigo que ande en caso de que la condición no se cumpla.

\begin{lstlisting}[language=c,style=customc,caption={Bloque Else}]
if(condicion){
    // Codigo
}
else{
    // Codigo
}
\end{lstlisting}

\begin{ejer}
    Modifique el codigo del programa \texttt{esPar} para que si recibe un numero no par imprima NO
\end{ejer}

Por ultimo, se puede agregar un \texttt{else if} para agregar una nueva condición que sea comprobada en caso que la primera sea falsa.

\begin{lstlisting}[language=c,style=customc,caption={Bloque else if}]
if(condicion){
    // Codigo
}
else if{
    // Codigo
}
\end{lstlisting}

Puede notar que esto es muy similar a lo ivsto en python.

\subsection{Switch}

El bloque \texttt{switch} permite cambiar una condición dependiendo del tipo de dato que se entregue. El switch cambiara de condición segun si es igual o no al dato. La syntaxis del \texttt{switch} es la siguiente.

\begin{lstlisting}[language=c,style=customc,caption={Bloque switch}]
switch(variable){
    case 1:
        // codigo
    case 2:
        // codigo
    ..
    case default:
        // codigo
}
\end{lstlisting}

\section{Ciclos}

\subsection{For}

\subsection{While}

\subsection{Do}

\section{Ejercicios Adicionales}