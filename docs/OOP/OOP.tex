OOP (Object Oriented Programming), o sus siglas en español, POO (Programación Orientada a Objeto) es una forma de programación enfocada en el uso de clases, herencias e interfaces, en donde entidades encapsulan metodos y datos, ademas del como interactuan con otros objetos, de tal manera que haya una estructura logica de relaciones. La POO es utilizada en varios ambitos, tales como inteligencia artificial y videojuegos.

En esta sección hablaremos sobre como realizar OOP en C++. OOP no es limitado a este lenguaje, ya que hay otros que lo implementan (tales como C\# o Java), sin embargo, la razon del por que en el curso nos movemos de C a C++ es por varias de las implementaciones de OOP que esta posee. C++ es considerado un lenguaje de alto nivel por esta razon, al poder permitir mas abstracción.

\section{Clases}
Una clase es un tipo de dato que definimos nosotros, en donde podemos asociarle atributos y metodos para que este objeto funcione, independiente de sus propiedades intrinsecas. En general, es mas facil entender la idea de clases mediante ejemplos, asi que haremos uno.

Una clase que podriamos considerar seria un auto.

\begin{itemize}
	\item Un auto tiene propiedades (tales como su patente, o la velocidad en la que va). Estas propiedades del auto se les conoce como los \textbf{atributos} de la \textbf{clase} auto.
	\item Un auto tiene funciones o metodos que nos permiten interactuar con el. Por ejemplo, una 'funcion' que podriamos realizar seria el acelerador, donde al pulsarlo hace que el auto acelere. Estas 'funciones' se les conoce como los \textbf{metodos} de la \textbf{clase} auto. Los \textbf{metodos} son funciones definidas dentro y para la clase.
	\item Podemos entender que cuando pensamos en un auto, la mayoria comparte las mismas propiedades (tener un patente, tener velocidad) y metodos (tener un acelerador). Al ter una \textbf{clase} auto, sabemos que tendra estas propiedades y metodos, dado que estan definidos dentro de su clase, por lo que nos asegura su funcionamiento
\end{itemize}

Con esto, podemos entonces tener un tipo de objeto el cual nos asegura que contenga los atributos y metodos definidios. Eso significa que si una variable es del tipo auto, tendra si o si el metodo 'acelerador' o un patente de atributo. Para definir la clase, se utiliza la palabra \texttt{class}.

\begin{lstlisting}[language=c,style=customc,caption={Definir clase}]
class Auto{ // Definición de la clase 'Auto'
	public:
		std::string patente; // Atributo de la clase
		int velocidad; // Atributo de la clase

		int acelerar(int nueva_velocidad){ // Metodo de la clase
			// codigo...
		}
};
\end{lstlisting}

Luego podemos definir distintos autos en base a las caracteristicas que queramos. Al ser la clase un tipo de auto, podemos definir variables que contenga los datos de auto, para luego definirle su patente, el motor, o usar metodos en el (Usar el acelerar para poder cambiar la velocidad). Cuando creamos una variable con estos valores, se le conoce como una \textbf{instancia} de la \textbf{clase} auto. En el siguiente codigo se muestra como definimos dos instancias distintas de distintos vehiculos.

\begin{lstlisting}[language=c,style=customc,caption={Definir clase}]
class Auto{ // Definición de la clase 'Auto'
	public:
		std::string patente; // Atributo de la clase
		int velocidad; // Atributo de la clase

		int acelerar(int nueva_velocidad){ // Metodo de la clase
			// codigo...
		}
};

int main(){
	Auto vehiculo1; // Instancia del vehiculo
	Auto vehiculo2; // Otra instancia del vehiculo

	vehiculo1.patente = 'GK-SB-82'
	vehiculo2.patente = 'HK-SK-25'

	// Ambas instancias comparten el atributo patente, pero al ser instancias distintas, cada una tiene sus propios valores

}
\end{lstlisting}

Para acceder a los metodos y atributos de una clase se utiliza el \texttt{.}, tal como aparece en el ejemplo superior. Notar que si creamos un puntero de una clase, la forma de acceder cambia del punto a una flecha \texttt{->}.

Con eso tenemos el funcionamiento basico de las clases. Debemos notar eso si una sección del codigo, la cual dice \texttt{public}.


\section{Modificadores de acceso}
La palabra \texttt{public} usada anteriormente no viene sin razon. Este es una forma de indicar los permisos de acceso para los elementos de una clase, conocidos formalmente como modificadores de acceso.

\subsection{Modificadores de acceso}

Un modificador de acceso permite bloquear el acceso de un atributo o metodo a utilizar. Existen tres principales formas de indicar acceso.

El primero corresponde al \texttt{public}. Public indica publico, por lo que no pone barreras al acceso de los datos. Tanto la propia clase como desde fuera se puede acceder y modificar estos datos. En el ejemplo que llevabamos trabajando previamente, pudimos acceder desde \texttt{main} al atributo de patente, al ser este publico

El segundo corresponde a \texttt{private}. Por como su nombre lo indica, es privado y unicamente puede ser accedido por su propia clase. Esto significa que si desde \texttt{main} intentamos accederlo, nos dara error. Para ejemplificarlo, hagamos que el parametro de velocidad de la clase Auto sea privado.

\begin{lstlisting}[language=cpp,style=customc,caption={Metodo de acceso private}]
class Auto{ // Definición de la clase 'Auto'
	private:
		int velocidad; // Atributo privado, unicamente accesible por la propia clase

	public:
		std::string patente; // Atributo publico, puede accederse desde fuera de la clase

		int acelerar(int nueva_velocidad){ // Metodo publico, puede accederse desde fuera de la clase
			// codigo...
		}
};
\end{lstlisting}

Primero, notar que dentro de una clase podemos tener mas de un modificador de acceso respecto a las necesidades que tengamos. En el ejemplo, el atributo velocidad se encuentra privado, mientras que patente y el metodo acelerar se encuentran aun en publico. Cuando accedimos a patente en \texttt{main} mediante el codigo \texttt{vehiculo1.patente} solo fui posible dado que el atributo es publico. Si quisieramos ahora acceder al atributo de velocidad desde \texttt{main}, dara error al no estar dentro de una clase.

Si queremos acceder a el, debemos hacerlo dentro de la clase. Una manera comun de hacerlo es mediante los metodos. Dado que tenemos un metodo llamado 'acelerar' que recibe como parametro \texttt{nueva_velocidad}, podemos entonces hacer que este metodo cambie la velocidad del vehiculo. Esto funciona dado que el metodo 'acelerar' esta definido dentro de la clase, y por tanto puede acceder a los atributos y metodos definidos con un modificador de acceso \texttt{private}.

\begin{ejer}
	Cree el codigo de la función \texttt{acelerar}, donde reciba de parametro una velocidad, y cambie el valor del atributo privado a la recibida en el metodo. Asegurese que retorne la velocidad privada, para asegurarnos de que haya camnbiado el valor interno del metodo.
\end{ejer}

Por ultimo, esta el modificador de acceso \texttt{private}, que oculta como al private, exceptuando a clases hijas de la actual. Esto es contenido de herencia que se vera despues, asi que por el momento definirlo como un private.

\subsection{Getters y Setters}

Al haber hecho que el funcionamiento de \texttt{acelerar} sea cambiar el valor del atributo privado \texttt{velocidad}, transformamos a este metodo en un tipo de metodo especial, llamado \textbf{setter}. 

La función de un metodo set (o setter) es la de modificar el valor de un atributo privado. De esta manera, si nos encontramos fuera de la clase, igual podremos modificar estos valores haciendo uso del metodo que es publico. Se le llama setter dado que cambia (set) el valor del atributo privado.

Si por otro lado quisieramos tener un metodo que nos permita obtener el valor de un atributo privado, entonces estariamos frente a un metodo \textbf{Get}. Se les llama get (o getter) dado que obtienen (get) el valor del atributo privado.

Estas funciones permiten mantener el estado del atributo privado, pero a la vez poder acceder y modificar sus valores si es necesarios. Ademas, como son metodos, podemos agregar condiciones para poder modificar los valores, modificarlo segun el estado de la clase, etc. En general, el formato de estos metodos es \texttt{set<Attribute>}
 o \texttt{get<Attribute>}, pero no es estrictamente necesario.

\begin{ejer}
	En la clase Auto, cambie el atributo de \texttt{patente} a privado, y cree dos metodos para esta, un set y un get, para poder modificar y ver el valor de la patente, respectivamente.
\end{ejer}

\subsection{Encapsulamiento}

Un concepto importante de conocer es el de \textbf{encapsulamiento}, el cual nos indica de mantener datos sensibles para el usuario oculto para ellos, para asi no tener mayores complicaciónes en su uso.

En general, estos dependeran del contexto. La mayoria de veces se tomara los datos personales o privados como encapsulamiento. Luego, hay formas de usar encapsulamiento para simplificar el uso de clases, tales como el TDA (Tipo de Datos Abstractos), en donde sabemos como interactuar con el TDA, pero no sabemos como funciona o que datos guarda internamente, al tenerlos encapsulados.

(Esta descripción es un poco corta para todo lo que refiere a TDA, pero se vera en su curso correspondiente de Estructura de Datos)


\section{Constructor Y Destructor}
En las definiciones basicas sobre OOP, definimos la instancia de una clase. Recordando el concepto, una clase puede tener varias instancias, eso significa, varias variables que comparten la estructura de la clase pero con datos distintos dependiendo de si misma. Siguiendo el ejemplo del Auto, dos instancias de un Auto pueden ser un Peugeot y un Ferrari.

Lo importante de las instancias es que cuando las creamos (Definidas por ejemplo dentro de una variable) los datos dentro de esta no vienen inicializados. Una manera de poder inicializar los valores al realizar una instancia es mediante el constructor.

\subsection{Constructor}

El constructor es un tipo especial de metodo el cual se ejecuta al crear la instancia de un objeto. Es decir, cuando creemos la instancia, este metodo se ejecutara, permitiendonos generar una base para los datos que usaremos. Un constructor se escribe de la forma \texttt{<Classname>()}, o sea, comparte el nombre de la clase.

Continuemos el ejemplo del Auto. En el auto, queremos que al iniciar una instancia de este, la velocidad inicial este en $0$. Un constructor que haga esto se veria de la siguiente manera

\begin{lstlisting}[language=c,style=customc,caption={Constructor de una clase}]
class Auto{ // Definicion de la clase 'Auto'
	private:
		int velocidad; // Atributo privado, unicamente accesible por la propia clase
		std::string patente; // Atributo privado, 

	public:

		Auto(){ // Constructor de la clase
			velocidad = 0; // Hara que al instanciarla la velocidad quede en 0.
		}

		std::string getPatente(){ // getter para el atributo patente
			return patente;
		}

		int acelerar(int nueva_velocidad){ // Metodo publico, puede accederse desde fuera de la clase
			// codigo...
		}
};
\end{lstlisting}

\begin{ejer}
	Modifique el codigo superior para que el constructor de auto cree una patente generica para el auto, asi entendiendo que no tiene patente de base.
\end{ejer}

Los constructores, al ser metodos, son funciones, y como son una funcion, permiten \textbf{sobrecarga} de funciones. Podemos entonces declarar varios constructores que funcionaran de manera distinta segun los parametros recibidos. Lo unico que hay que hacer es redeclarar el metodo constructor con distintos parametros. Por ejemplo, asi se veria una sobrecarga de un constructor si queremos que el auto inicie con una velocidad especifica.

\begin{lstlisting}[language=C++,style=customc,caption={Sobrecarga de Constructor}]
class Auto{ // Definicion de la clase 'Auto'
	private:
		int velocidad; // Atributo privado, unicamente accesible por la propia clase
		std::string patente; // Atributo privado, 

	public:

		Auto(){ // Constructor de la clase
			velocidad = 0; // Hara que al instanciarla la velocidad quede en 0.
		}

		Auto(int velocidad_inicial){ // Sobrecarga de Constructor. Se le entrega parametro int
			velocidad = velocidad inicial; 
		}

		// Mas codigo
};

int main(){
	Auto vehiculo1(100); // Este auto empezara con una velocidad de 100. (Sobrecarga de Constructor)
	Auto vehiculo2; // Este auto empezara con una velocidad de 0 (Constructor)
	return 0;
}

\end{lstlisting}

\begin{ejer}
	Modifique el codigo superior para que tenga un metodo constructor que reciba una patente y la asigne al parametro patente, en vez de asignarle una generica.
\end{ejer}

\subsection{Destructores}

Tanto como tenemos un metodo especial que se ejecuta cada vez que se crea una instancia de un objeto, tenemos un metodo especial que se ejecuta cada vez que se elimine y/o deje de usar una instancia, llamados destructores. Los destructores se escriben de la misma forma que un constructor, pero agregando un \texttt{~}. La forma generica de un destructor es \texttt{~<Classname>()}.

En C++, las clases tienen destructores propios. No hay necesidad de usar destructores en todas las clases. Los destructores deben usarse cuando tenemos punteros y/o memoria dinamica que se inicia dentro de la clase, puesto que si no se elimina tendremos problemas de \texttt{memory leak}.

Consideremos que tenemos un parametro \texttt{std::string pasajeros}, que tiene una arreglo dinamico con los nombres de los pasajeros del Auto. Un constructor se veria de la siguiente manera.

\begin{lstlisting}[language=C++,style=customCpp,caption={Sobrecarga de Constructor}]
class Auto{ // Definicion de la clase 'Auto'
	private:
		int velocidad; // Atributo privado, unicamente accesible por la propia clase
		std::string patente; // Atributo privado
		std::string* pasajeros; 

	public:

		Auto(){ // Constructor de la clase
			velocidad = 0; // Hara que al instanciarla la velocidad quede en 0.
			patente = "XX-XX-XX";
			pasajeros = new std::string(4); // Un arreglo dinamico de 4 pasajeros
		}

		~Auto(){ // Destructor de la clase
			delete[] pasajeros // Elimina la memoria dinamica creada en el parametro pasajeros.
		}

		// Mas codigo
};

int main(){
	return 0;
}

\subsection{Big Three}

La regla de los tres (Big Three Rule) consiste en una regla de implementaciones en clases de C++. Si una clase requiere de un destructor implementado, probablemente necesitara tambien una implementación de un constructor para copias y un operador de copias.

¿Para que necesitamos un operador de copia? Considerando que una clase tenga elementos que se les asignan recursos (tales como un puntero para memoria dinamica), si realizamos una copia de un objeto de esta clase en la forma \texttt{obj1 = obj2} hara que se copien los elementos exceptos aquellos que se les asigne recursos, siendo entonces una copia erronea del elemento. Por ello, habra que implementar un constructor de copias y el operador de copias.

\subsubsection{Constructor para copias}

El constructor para copias permite que al hacer una copia de un objeto, el puntero tenga asignada la misma ubicación de memoria que el objeto original, asi comparten elementos.

Si consideramos el codigo del Auto, una implementación del constructor para copias seria de la siguiente manera:

\begin{lstlisting}[language=C++,style=customCpp,caption={Constructor de copias}]
class Auto{ // Definicion de la clase 'Auto'
	private:
		std::string* pasajeros; 
		// Otros parametros

	public:

		// Definición de constructor...

		std::string* getPasajeros(){
			return pasajeros;
		}

		// Definición del Destructor
		~Auto(){ // Destructor de la clase
			delete[] pasajeros // Elimina la memoria dinamica creada en el parametro pasajeros.
		}

		// Constructor de copias
		Auto(const Auto& originalAuto){ // Recibe como parametro el auto a copiar
			pasajeros = originalAuto.getPasajeros(); // El pasajeros del o
		}

		// Mas codigo
};

int main(){
	// Asumir que vehiculo1 ya tiene un arreglo de pasajeros
	Auto vehiculo2(vehiculo1) // vehiculo2 tendra la misma lista de pasajeros que vehiculo1
	return 0;
}
\end{lstlisting}

Esta implementación hace que cuando \texttt{vehiculo2} usa la sobrecarga del constructor de copias, este se encarge de copiar los elementos originales del parametro indicado (Objeto/Instancia a copiar).

\subsubsection{Implementación de operador de copia}

Si quiere hacerse que se copien los elementos de una clase al hacer \texttt{obj1 = obj2}, se puede implementar mediante la redefinición del operador \texttt{=}. Esto se hace de la siguiente manera:

Nota: Modificar esto cuando encuentre un mejor ejemplo para el operador de copia

\begin{lstlisting}[language=C++,style=customCpp,caption={Implementación de operador de copia}]
class Auto{ // Definicion de la clase 'Auto'
	private:
		std::string* pasajeros; 
		// Otros parametros

	public:

		// Definición de constructor...
		

		std::string* getPasajeros(){
			return pasajeros;
		}

		// Definición del Destructor
		~Auto(){ // Destructor de la clase
			delete[] pasajeros // Elimina la memoria dinamica creada en el parametro pasajeros.
		}

		// Implementación de operador de copia
		Auto& operator=(const Auto& autoOriginal){
			
		}

		// Mas codigo
};

int main(){
	// Asumir que vehiculo1 ya tiene un arreglo de pasajeros
	Auto vehiculo2(vehiculo1) // vehiculo2 tendra la misma lista de pasajeros que vehiculo1
	return 0;
}
\end{lstlisting}





\section{Herencia}
Un ultimo concepto importante de OOP es la herencia. La herencia nos permite reutilizar codigo de una clase padre y darselos a una clase hijo. La clase hijo podra definir sus propios metodos y atributos, pero ademas obtendra todos aquellos del padre. Sirve para hacer un cierto nivel de abstracción en el desarrollo.

Considerando el ejemplo del Auto, podemos imaginar que esta pueda pertencer a una clase superior llamada Vehiculo. La clase Vehiculo contiene todos los elementos que deberia tener un vehiculo, mientras que la clase Auto define los propios del auto. Este ejemplo se veria de la siguiente manera:

\begin{lstlisting}[language=C++,style=customCpp,caption={Herencia en C++}]

// Clase padre que contiene los metodos generales de un vehiculo
class Vehiculo{
	protected:
		int ruedas;
		std::string patente;
		int velocidad;

	public:
		// Constructor y destructor...

		std::string getPatente(){
			return patente;
		}

		int acelerar(); // Declaración de función acelerar
}

class Auto : public Vehiculo{
	private:
		std::string; // Modelo del auto

	public:
		// Constructor y destructor

		int acelerar(); // Declaración de función acelerar

		void subirVidrio(); // Declaración de función subirVidrio
		// Notar como este metodo es unico de Auto, pero no de Vehiculo
}

class Moto : public Vehiculo{
	private:
		std::string; // Modelo del auto
	
	public:
		// Notar que hereda los mismos elementos que el padre, pero no de Auto
		// Naturalmente, no tiene la función de subirVidrio, dado que una Moto no tiene
}
\end{lstlisting}

En este ejemplo, la clase Auto hereda los elementos de la clase Vehiculo. Eso significa que si dentro de la función \texttt{acelerar} deseamos usar el parametro de velocidad, este estara declarado dado que proviene de su clase padre, Vehiculo. Lo mismo si intentamos utilizar el metodo \texttt{getPatente}.

Los parametros y metodos de la clase padre no se heredan automaticamente al hijo. Estos dependen de los metodos de acceso previamente definidos.
\begin{itemize}
	\item Los elementos \texttt{public} se heredan al hijo.
	\item Los elementos \texttt{private} se heredan al hijo, los tendra, pero no podra acceder a ellos.
	\item Los elementos \texttt{protected} se heredan al hijo, la clase hija podra acceder a estos contenidos, pero no pueden ser accedidos fuera de la clase.
\end{itemize}

%% Debo aun incluir lo de las clases genericas.


