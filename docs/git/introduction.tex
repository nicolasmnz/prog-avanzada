\subsection{git}


Como ya mencionamos, Git es un sistema de control de versiones. Eso significa que al crear un repositorio tendremos todo un historial de los cambios que vamos realizando en nuestro repositorio.

Para iniciar un repositorio se debe tener git instalado. Luego, en la carpeta donde queramos ir guardando los cambios se debe utilizar el comando \texttt{git init}.

\begin{lstlisting}[language=bash, style=terminal, caption={git init: Iniciar Repositorio Git}]
	user@hostname:~\$ git init
\end{lstlisting}

Recomendamos para la siguiente sesiones tener un repositorio (O crear uno falso) para ir probando todos los comandos que sirven para control del repositorio. En este caso, considere un repositorio llamado 'falso-repositorio', que utilizaremos para el desarrollo. Inicie y cree un repositorio con el.
