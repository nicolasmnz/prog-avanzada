%% Git es un sistema de control de versiones open-source que nos permite trabajar con historial de cambios dentro de repositorios personales o colaborativos. La mayoria de la gente conoce mas GitHub que Git, siendo que en estricto rigor, github nace de git. 


%$En este laboratorio aprenderemos el uso de git basico que permita aprender a utilizar los repositorios de una forma eficiente y productiva. Git no sera solo contenido de curso, si no que probablemente algo que les funcionara para toda su carrera.

%% Para esta sección, recomendamos saber utilizar al menos los comandos de terminal, ya que este laboratorio sera desarrollado la mayoría desde esta. Ademas, notar que los ejercicios de esta sección estan mas orientados a que sepan como utilizar los comandos y tengan un repositorio al final.

Git es un sistema de control de versiones open-source diseñado para poder gestionar proyectos de codigo con velocidad y eficiencia.

Esta es una guia para poder controlar un repositorio de manera eficiente. Para esto, puedes leer esta guia o utilizar otros medios de aprendizaje que recomendamos en su sección respectiva.

\section{Introducción a git}
\subsection{git}


Como ya mencionamos, Git es un sistema de control de versiones. Eso significa que al crear un repositorio tendremos todo un historial de los cambios que vamos realizando en nuestro repositorio.

Para iniciar un repositorio se debe tener git instalado. Luego, en la carpeta donde queramos ir guardando los cambios se debe utilizar el comando \texttt{git init}.

\begin{lstlisting}[language=bash, style=terminal, caption={git init: Iniciar Repositorio Git}]
	user@hostname:~\$ git init
\end{lstlisting}

Recomendamos para la siguiente sesiones tener un repositorio (O crear uno falso) para ir probando todos los comandos que sirven para control del repositorio. En este caso, considere un repositorio llamado 'falso-repositorio', que utilizaremos para el desarrollo. Inicie y cree un repositorio con el.

\section{Commits}
Inicialmente nuestro repositorio se encuentra vacio. Todos los elementos que vayas creando van a mantenerse, pero el repositorio no contendra estos cambios hasta que le digas manualmente que cambios has realizado, y cuales quieres guardar.

Un archivo comun dentro de la mayoria de repositorios es aquel conocido como \textbf{README.md}, que contendra información basica del repostiorio para gente que quiera utilizarlo.

Tras crear el archivo, debemos decirle a git que tenemos la intención de agregar el archivo. Para hacer esto, podemos usar el comando \texttt{git add}.

\begin{lstlisting}[language=bash, style=terminal, caption={git add: Agregar un archivo especifico a la lista de cambios}]
	user@hostname:~\$ git add <filename>
\end{lstlisting}

De este modo, git sabra que queremos agregar el README como un cambio. Este cambio aun no se encuentra confirmado, pero en un estado de espera. Puedes agregar mas cambios que se suban usando el mismo comando. Si quieres agregar todos los cambios que han ocurrido en el repostiorio, se puede \texttt{git add} con una flag distinta.

\begin{lstlisting}[language=bash, style=terminal, caption={git add: Agregar todos los cambios}]
	user@hostname:~\$ git add .
\end{lstlisting}

Cuando tengamos listo todos los cambios listos que queriamos subir agregados, se deben confirmar y subir al repositorio. Este proceso se le conoce como \textbf{commit}. Al hacer commit, git guardara los cambios realizados desde el anterior commit al que acabamos de realizar. Podriamos decir que cuando realizamos el comando de \texttt{git init} se produjo un commit inicial del sistema, en donde agrega todos los archivos iniciales como cambios. Para hacer \texttt{git commit}, utilizar el comando en la terminal.

\begin{lstlisting}[language=bash, style=terminal, caption={git commit: Subir cambios al repositorio}]
	user@hostname:~\$ git commit
\end{lstlisting}

Cuando se utilize el comando se abrira una ventana en donde mostrara los cambios realizados. En la parte superior se puede escribir un mensaje que describa los cambios realizados. Este mensaje puede ser cualquiera, pero recomendamos utilizar un estandar.


\section{Branches}
\subsection{Introducción a las branch}

Una caracteristica importante de git son las \textbf{Branch}. Una branch se considera una rama del trabajo que estamos realizando. 
Cuando creamos el repostiorio usando \texttt{git init}, en realidad creamos la branch principal que se conoce como \texttt{main} o \texttt{master}. Esta branch se considera la principal, en donde se encuentran todos los cambios realizados o en general funcionales. 

Para poder explicar el funcionamiento de las branches utilizaremos imagenes de una pagina llamada \href{https://learngitbranching.js.org/?NODEMO}{learngitbranching}. Tambien pueden aprender desde esta fuente.

La siguiente imagen muestra el inicio del repositorio tras usar git init (Commit 0, C0) y al subir el README (Commit 1, C1).

\begin{figure}[H]
    \centering
    \includegraphics[width=0.5\linewidth]{img/Branch Main.png}
    \caption{Branch Main}
\end{figure}

Cuando creemos una branch, esta hara una rama desde el commit desde que nos encontramos (En este caso, C1, que es el ultimo cambio que subimos. Hay formas de moverse entre commits para hacer ramas en previos commits, pero no se veran en esta guia). Para hacer la branch, utilizar el comando \texttt{git branch}

\begin{lstlisting}[language=bash, style=terminal, caption={git branch: Crear una nueva branch}]
    user@hostname:~\$ git branch <branch-name>
\end{lstlisting}

El resultado de este comando se ve en la siguiente imagen.

\begin{figure}[H]
    \centering
    \includegraphics[width=0.5\linewidth]{img/new_branch.png}
    \caption{Nueva Branch: Feature}
\end{figure}

En este ejemplo se creo una branch con el nombre 'feature'. En general, las branches son utilizadas para agregar un cambio especifico que se debe desarrollar o probar, y que no queremos que modifique la branch principal hasta que este todo listo y correcto para funcionar.

Como ahora tenemos dos ramas distintas. Los cambios (commits) realizados dentro de la branch 'feature' no seran vistos dentro de la branch 'main' y viceversa. Podran desarrollarse de manera independiente. Si queremos movernos entre branches se utiliza el comando \texttt{git checkout}.

\begin{lstlisting}[language=bash, style=terminal, caption={git checkout: Moverse a una branch}]
    user@hostname:~\$ git checkout <branch-name>
\end{lstlisting}

\subsection{Juntar cambios entre branches}

Cuando tengamos todos los cambios realizados que queremos en la nueva branch y creemos que ya es momento de integrarlos en 'main', podremos utilizar dos metodos distintos de integrar los cambios. Consideremos que queremos insertar los cambios hechos en 'feature' a 'main'.

\subsubsection{git merge}

\texttt{git merge} permite hacer un commit en la branch que recibira los cambios. En este commit, se incluyen todos los cambios que se tienen de la otra branch con la branch actual, ademas de la información de cuando se hizo el merge, desde que branch y hacia cual. Este tipo de commits se le conoce como el 'merge commit'. 

\begin{lstlisting}[language=bash, style=terminal, caption={git merge: Fusionar la branch actual con otra branch}]
    user@hostname:~\$ git merge <branch-name>
\end{lstlisting}

El resultado se ve en la siguiente imagen:

\begin{figure}[H]
    \centering
    \includegraphics[width=0.5\linewidth]{img/merge.png}
    \caption{Resultado de git merge}
\end{figure}

\subsubsection{git rebase}

\texttt{git rebase} no realiza un 'merge commit', si no que toma todos los commits realizados en la branch actual y los realiza en la branch que recibira los cambios. Esto hace que parezca que nunca hubo una branch. Para hacer esto, utilizar el siguiente comando.

\begin{lstlisting}[language=bash, style=terminal, caption={git rebase: Hacer rebase de la branch actual hacia otra}]
    user@hostname:~\$ git rebase <branch-destino>
\end{lstlisting}

Los cambios se ven de la siguiente manera:

\begin{figure}[H]
    \centering
    \includegraphics[width=0.5\linewidth]{img/rebase.png}
    \caption{Resultado de git rebase}
\end{figure}

Ambas opciones son igual de validas. Unicamente cambiaran la forma en que se ve la branch, pero los resultados dentro del contenido del repositorio son equivalentes.

\subsection{Eliminar la branch}

Cuando ya tengamos los cambios de una branch integrados dentro de la principal, lo mas probable es que esta branch ya no tenga mas sentido mantenerla. Podemos eliminar del repositorio utilizando la flag \textit{-d}, que indica eliminación de la branch.

\begin{lstlisting}[language=bash, style=terminal, caption={git branch: Borrar branch}]
    user@hostname:~\$ git branch -d <branch>
\end{lstlisting}

Si hay cambios dentro de la branch que no se guardaran al eliminarse, le saldra un mensaje de advertencia. En caso de que de verdad quiera borrar estos cambios, cambie la flag a \textit{-D}

Con estos comandos, ya se tiene lo necesario para utilizar git de manera correcta


\section{Git Remoto (Github)}

El mayor exponente, y por lo mas probable que conociste git, es Github. Notar que Github es una plataforma para poder mantener, publicar, y compartir repositorios en la web, mas no es el unico que utiliza git para mantener su plataforma.

Los comandos que utilizaremos ahora nos permitiran hacer control de los cambios y traspasarlos al repostorio en linea. Utilizaremos la plataforma de Github, pero esto puede extrapolarse a otras plataformas que utilizen git.

\subsection{Clonar un repositorio}

Podemos clonar un repositorio de la siguiente manera. Primero, moverse en la terminal al directorio donde queremos que el repositorio sea copiado. Cuando lo clonemos, se creara una carpeta con todo el contenido del repositorio. 

Cuando ya lo tengamos, clonamos con el siguiente comando:

\begin{lstlisting}[language=bash, style=terminal, caption={git clone: Clonar un repositorio}]
    user@hostname:~\$ git clone <url-repositorio>
\end{lstlisting}

Cuando clonemos el repositorio, tendremos todos los contenidos de este en una carpeta con el mismo nombre. Al clonar, tendremos una copia de todos los estados del repositorio, incluyendo branches y commits.

Hay que explicar un concepto importante y es sobre la branch 'origin'. Si realizamos el comando \texttt{git branch -a}, podremos ver una lista de todas las branches que tiene el repositorio. Aquellas que tienen el formato 'origin/<branch-name>' son branches que se encuentran en el repositorio remoto, e indican el estado de este. Por si solas, estas branches no son modificables, para poder realizar cambios en la branch, debemos crear una branch en el formato '<branch-name>' (Es decir, que tenga el mismo nombre pero sin 'origin') que sera nuestra copia local del estado de esa branch. Los commits que realizemos dentro de la branch local no se subiran inmediatamente al repositorio, si no que debemos usar el comando \texttt{git pull} para que se vea cambiado el repositorio remoto.
\begin{lstlisting}[language=bash, style=terminal, caption={git push: Subir los cambios locales a la branch remota}]
    user@hostname:~\$ git push
\end{lstlisting}

Si en cambio queremos actualizar nuestra branch local en caso de que hayan habido cambios en la branch remota, utilizamos el comando \texttt{git pull}

\begin{lstlisting}[language=bash, style=terminal, caption={git pull: Recibir cambios realizados en la branch remota}]
    user@hostname:~\$ git pull
\end{lstlisting}

De esta forma, podremos ir trabajando tanto en nuestra maquina como en el repositorio remoto.

Hay un caso importante que mencionar aqui. Si tenemos una branch local que no se encuentra en el repositorio remoto, podemos usar el comando \texttt{git push} con el nombre de la branch para que esta sea enviada al repositorio remoto.

\begin{lstlisting}[language=bash, style=terminal, caption={git push: Subir branch al repositorio remoto}]
    user@hostname:~\$ git push -u <branch-name>
\end{lstlisting}

La flag \texttt{-u} indica a git que los cambios realizados en la branch local pueden ser subidos a su correspondiente branch remota. Es equivalente a la \texttt{--set-upstream}, pero no nos referiremos a este aqui.


\section{Fuentes y mas}

\begin{itemize}
	\item \href{https://learngitbranching.js.org/?NODEMO}{learngitbranching}
\end{itemize}

\section{Resolución de Problemas}

En el mejor de los casos, no habra errores que arreglar dentro de estos commits. En la practica es posible que ocurran momentos en donde haya problemas entre versiones. Los problemas mas comunes que encontramos varian, pero aqui denotamos la solución a varios de ellos.

\subsection{Ver un commit anterior}

Consideremos un estado de un repositorio en el que tenemos tres commits, \texttt{C1}, \texttt{C2} y \texttt{C3}. \texttt{HEAD} se encuentra en C3. 

Digamos que queremos volver a C1 debido que habia una caracteristica que queriamos revisar. Para volver a un commit anterior para observarlo puede usar el comando \textbf{git switch}.

Primero necesitamos saber el id de los commits, que se le conoce como hash. Puede saber esta informacióin mediante \textbf{git log}.

\begin{lstlisting}[language=bash, style=terminal, caption={git log: Ver log del repositorio}]
    user@hostname:~$ git log
\end{lstlisting}

Teniendo el id de este commit, podemos ver el commit utilizando este id.

\begin{lstlisting}[language=bash, style=terminal, caption={git switch: Cambiar a commit}]
    user@hostname:~$ git switch <hash-commit>
\end{lstlisting}

Esto hara que nuestra ubicación actual (o sea, el \texttt{HEAD}) se mueva hacia otra posición. Esto nos permite ver el estado de aquel commit. Para volver al commit principal de la branch actual, utilice el siguiente comando.

\begin{lstlisting}[language=bash, style=terminal, caption={git switch: Volver al ultimo commit de la branch}]
    user@hostname:~$ git switch <branch>
\end{lstlisting}

\subsection{Rehacer un commit anterior}

Digamos estamos en el commit \texttt{C3}, y queremos volver al estado del commit C2. Para esto, tenemos varios comandos. Uno de ellos es \textbf{git revert}, que permite quitar los cambios realizados en un commit. Esto lo hace creando un commit nuevo que borra el commit indicado.

\begin{lstlisting}[language=bash, style=terminal, caption={git revert: Quitar modificaciones de un commit}]
    user@hostname:~$ git revert <hash-commit>
\end{lstlisting}

Otra manera de volver a un estado anterior es utilizando \textbf{git reset}. Este comando, dependiendo de la flag, puede modificar el historial del repositorio, por lo que no lo recomendamos. Puede investigar sobre el usando el comando man.

\subsection{Merge Conflicts}

Aveces cuando juntemos dos branches de repositorios, estas pueden contener archivos que contengan distintas versiones entre si. Esto se conoce como un \texttt{MERGE CONFLICT}.

Dado que le aparezca merge conflict, puede utilizar el comando \textbf{git status} para saber el estado actual.

\begin{lstlisting}[language=bash, style=terminal, caption={git revert: Quitar modificaciones de un commit}]
    user@hostname:~$ git status 
\end{lstlisting}

Aqui podremos ver el estado de error. Simulemos un caso.

\begin{ejer}
    Cree un repositorio en el cual primero haga el commit de un \texttt{README.md}. Despues haga una branch llamada \texttt{Conflict} y agrege como titulo al readme \texttt{\#Conflict}. En su branch principal, agregue de titulo \texttt{\#Main}. Haga un merge y observe el estado usando git status.
\end{ejer}

En el archivo conflictivo que le aparezca en git status, vera que se escribiran formatos de la siguiente manera...

\begin{lstlisting}[language=bash]
    <<<<<<<<<<HEAD
    #Main
    ==========
    #Conflict
    >>>>>>>>> feature
\end{lstlisting}

En este caso, el conflicto esta en la primera linea, en donde la branch actual toma valor de Main, y en la branch feature toma valor de conflict. Para solucionarlo, debe borrar tanto las lineas con $<<<$ y $>>>$, como la linea con $===$, y debe escoger entre cual de las dos soluciones se queda. Tambien puede hacer una nueva, si lo ve necesario.

\begin{ejer}
    Arregle el error del ejercicio anterior. Hagale commit y subalo.
\end{ejer}

\section{Git Remoto}

Finalmente, veremos el trabajo colaborativo de git. Todo este formato de versiones podemos subirlo a otras aplicaciones online que mantengan una copia del estado.

Miremos por ejemplo Github, la cual es la mayormente conocida dentro de este ambito. 

\section{Commits Convencionales}

Ya se tiene conocimiento de los commits. Sin embargo, entregare aqui varios ejemplos de como usar los commits para que nos apoyen mas que hacernos perder tiempo.

\begin{enumerate}
    \item Intenta siempre hacer commits que indiquen bien que hacen. Por ejemplo, puede utilizar \href{https://www.conventionalcommits.org/en/v1.0.0/}{Conventional Commits}
    \item Intente hacer branches que no duren mucho para evitar cambios fuertes entre main y branch. Haga branches enfocadas en una caracteristica especifica, hagale merge, y borre la branch.
\end{enumerate}
